Mängden radioaktiv strålning som återfinns i livsmedel har undersökts. Bland
proverna har ingått både vildmarkslivsmedel och det tillverkade mineralsaltet
Seltin. Dessutom undersöks $\beta$- och $\gamma$-sönderfall, samt inre
konversion.

För att fastställa aktivitetsnivåerna hos Cesium-137 (Cs-137) och hos Kalium-40
(K-40) användes både en scintillationsdetektor och en halvledardetektor.
Isotopen Cs-137 har i naturen sitt ursprung i Tjernobylolyckan år 1986, medan
K-40 förekommer naturligt.

Vid mätning av radioaktiva isotoper i svamp från Kiev år 2002 fanns
aktiviteten från Cs-137 till \qty{18000}{\becquerel\per\kg}. Detta
överskrider Livsmedelsverkets gränsvärden på \qty{1500}{\becquerel\per\kg} för
vildmarkslivsmedel med stor marginal. Vidare uppgick K-40-aktiviteten i
Seltinsaltet till \qty{6300}{\becquerel\per\kg}, vilket är långt över
Livsmedelsverkets gräns på \qty{300}{\becquerel\per\kg}.

Mätningen av betastrålning från Cs-137 visade ett klart definierat
energispektrum, där topparna från inre konversion visade sig tydligt. För att
kalibrera mätutrustningen användes K-toppens teoretiska energivärde. L-toppen
uppmättes då till \qty{0.657}{\MeV}, att jämföra med det teoretiska värdet på
\qty{0.6564}{\MeV}. Genom en grafisk kurvanpassning över spektrumet bestämdes
den maximala sönderfallsenergin för Cs-137 till \qty{0.59}{\MeV}, att jämföra
med det teoretiska värdet \qty{0.513889}{\MeV}.
