\section{Inledning} \label{sec:introduction}

\subsection{Syfte} \label{sec:purpose}

Syftet med laborationen är att undersöka mängden aktiva atomer i
livsmedelsprover för att jämföra uppmätta värden med fastställda gränsvärden.
Vidare ska ett betaspektrum för Cs-137 studeras för att fördjupa förståelse
av dess sönderfalssprocesser och de fysikaliska principerna bakom aktiviteten.

\subsection{Bakgrund} \label{sec:background}

Radioaktivitet (eller bara ``aktivitet'') upptäcktes år 1896 av fysikern
Henri Becquerel, och forskades vidare på av Marie Curie~\cite{curie}. Det
var ren slump att Becquerel upptäckte att uransalt sände ut strålning som
svärtade en fotoplåt~\cite{milestones}. När aktivitet först upptäckted
visste man inte att strålningen var farlig då det inte fanns någon tidigare
forskning att indikera detta. Det ledde till att Curie och hennes kollegor
utsattes för stora mängder strålning genom sitt forskningsarbete. Forskningen
från Marie Curie och hennes make Pierre Curie, samt Becquerels upptäckt, ledde
till delat nobelpris i fysik år 1903.

Aktivitet är ett tillstånd för vissa instabila nuklider där de spontant kan
sönderfalla och därmed släppa ut radioaktiv strålning. Typen av strålning beror
på typen av sönderfall, bland dessa $\alpha$-, $\beta$- och
$\gamma$-sönderfall~\cite{arpansa}. Dessa strålningstypers unika egenskaper har
lett till flera användningsområden idag, exempelvis medicinska syften såsom
cancerbehandling men även som energikälla i form av kärnkraft.

Kärnkraftverk omsätter stora mängder energi, vilket medför stora risker. Ett
välkänt exempel på när kärnkraft resulterade i katastrof är Tjernobylolyckan år
1986. Olyckan ledde till att radioaktivt avfall som Cs-137 och I-131 släpptes
ut i naturen genom regn som förde med sig avfallet från atmosfäre till marken.
Än idag syns aktivitet från Cs-137 i livsmedel åd halveringstiden är kort
över 30 år~\cite{fysika}.

\subsection{Bakgrundsstrålning} \label{sec:backgroundrad}

Bakgrundsstrålning är den mängden ``passiv'' strålning som återfinns överallt i
naturen. Den kommer delvis från rymden i form av \textit{kosmisk}
bakgrundssstrålning, dels från ämnen i vår direkta omgivning såsom uran och
torium i marken. Konstgjorda källor av aktivitet bidrar också till
bakgrundsstrålningen, såsom Tjernobylolyckan och rester av kärnvapentexst.

Laborationen har utförts i ett laboratorium där aktiva ämnen och relaterad
utrustning används regelbundet. Det innebär att bakgrundsstrålningen troligen
lokalt är betydligt högre än vanliga nivåer. Den kanske främsta källan är
detektorn (se avsnitt~\ref{sec:method}) som är gjord av bly och innehåller
den instabila nukliden Pb-210.

\subsection{Val av prover} \label{sec:samples}

För undersökning har tre prover valts: två livsmedelsprover och en stav vars 
sammansättning är känd:

\begin{itemize}
    \item Svamp, plockad nära Kiev år 2002, som förväntas innehålla någon
    andel Cs-137 från Tjernobylolyckan \parencite{instructions}. Som visas i
    avsnitt~\ref{sec:theory} innehåller denna sönderfallskedja både
    $\gamma$- och $\beta^-$-sönderfall. Här har $\gamma$-sönderfallen
    undersökts.

    \item Mineralsaltet Seltin innehåller naturligt Kalium och har därmed
    någon aktivitet från innehållet K-40, som visat i bilaga~\ref{sec:seltin}.
    Även K-40 kan genomgå $\gamma$-sönderfall; detta har undersökts.

    \item En stav som innehåller någon mängd Cs-137. För denna har istället
    $\beta$-sönderfallet undersökts.
\end{itemize}

Svamp är valt för undersökningen av aktivitet i näringskedjan för dess förmåga
att absorbera och binda näring, och därmed även de aktiva atomerna, i marken.
Även andra växter eller vissa djurprodukter kan väljas i detta syfte.

Cs-137 från Tjernobylolyckan återfinns i livsmedel som produceras än idag,
speciellt i områden var mer avfall nådde som följd av dåvarande väderlek. Även
Seltin säljs idag. Att undersöka aktiviteten i dessa vanliga livsmedel är av
intresse för vardagligt bruk då det är en fråga om Folkhälsa
(se avsnitt~\ref{sec:conclusions}).
