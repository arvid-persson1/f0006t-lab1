\section{Teoretisk beräkning av aktivitet i Seltin} \label{sec:seltin}

Hälsosaltet Seltin består till massan \qty{21}{\percent} av naturligt Kalium.
Aktiviteten i \qty{66.0}{\g} Seltin som följd av K-40-innehållet ska beräknas.

\subsection*{Indata}

\subsubsection*{Givet}

\begin{align*}
    & \text{Massa Seltin}          & m_S = \qty{0.066}{\kg} \\
    & \text{Andel Kalium i Seltin} & r_S = \qty{21}{\percent}
\end{align*}

\subsubsection*{Konstanter}

\begin{align*}
    & \text{Andel K-40 i naturligt Kalium} & r_K     = \qty{0.0117}{\percent}            &\quad \fysika{Tk4} \\
    & \text{Nuklidmassa K-40}              & M       = \qty{39.9639982}{\atomicmassunit} &\quad \fysika{Tk4} \\
    & \text{Halveringstid K-40}            & T_{1/2} = \qty{1.248}{\giga\scyear}         &\quad \fysika{Tk4}
\end{align*}

\subsection*{Enhetsomvandlingar}

\begin{align*}
    & \qty{1}{\scyear} &= \qty{31557600}{\s}                &\quad \fysika{Te} \\
    & \qty{1}{\atomicmassunit} &= \qty{1.66053907e-27}{\kg} &\quad \fysika{Te}
\end{align*}

\subsection*{Formler}

\begin{align}
    M       &= \frac{m}{N}           &\quad \fysika{Fa2} \label{eq:molmass} \\
    A       &= \lambda N             &\quad \eqref{eq:activity} \nonumber   \\
    T_{1/2} &= \frac{\ln 2}{\lambda} &\quad \eqref{eq:halflife} \nonumber
\end{align}

\subsection*{Lösning}

Mängden kalium $m_K$ är andelen kalium i Seltin gånger massan Seltin;
%
\begin{equation}
    m_K = m_S r_S~\text{(\unit{\kg})} \label{eq:kmass}
\end{equation}

Mängden K-40 $m$ är andelen K-40 i naturligt kalium gånger massan Kalium;
%
\begin{equation}
    m = m_K r_K~\text{(\unit{\kg})} \label{eq:k40mass}
\end{equation}

Av \eqref{eq:molmass} fås substansmängden $N$:
%
\begin{equation}
    M = \frac{m}{N} \iff N = \frac{m}{M} \label{eq:substance2}
\end{equation}

Av \eqref{eq:halflife} fås sönderfallskonstanten $\lambda$:
%
\begin{equation}
    T_{1/2} = \frac{\ln 2}{\lambda} \iff \lambda = \frac{\ln 2}{T_{1/2}}~\text{(\unit{\per\s})} \label{eq:decay}
\end{equation}

\eqref{eq:decay} och \eqref{eq:substance2}, sen \eqref{eq:k40mass}, sen
\eqref{eq:kmass} i \eqref{eq:activity} ger aktiviteten $A$:
%
\begin{equation}
    A = \lambda N = \frac{m_S r_S r_K \ln 2}{M T_{1/2}} \approx \qty{4.3e2}{\becquerel}
\end{equation}

Resultat: \qty{4.3e2}{\becquerel}.
