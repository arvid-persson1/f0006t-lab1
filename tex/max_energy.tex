\section{Maximal elektronenergi vid Cs-137-sönderfall} \label{sec:maxenergy}

Elektronernas maxenergi för de båda $\beta^-$-sönderfallen av Cs-137 (se
avsnitt~\ref{sec:theory}) ska beräknas.

\subsection*{Konstanter}

\begin{align*}
    & \text{Ljusets hastighet}        & c        = \qty{299792458}{\m\per\s}         &\quad \fysika{Te}     \\
    & \text{Nuklidmassa Cs-137}       & M_{Cs}   = \qty{136.907089}{\atomicmassunit} &\quad \fysika{Tk4}    \\
    & \text{Nuklidmassa Ba-137}       & M_{Ba}   = \qty{136.905827}{\atomicmassunit} &\quad \fysika{Tk4}    \\
    & \text{Excitationsenergi Ba-137} & E_\gamma = \qty{0.661657}{\MeV}              &\quad \text{(avsnitt~\ref{sec:theory})}
\end{align*}

\subsection*{Enhetsomvandlingar}

\begin{align*}
    & \qty{1}{\atomicmassunit} = \qty{931.494103}{\MeV\per\clight\squared} \quad \fysika{Te}
\end{align*}

\subsection*{Formler}

\begin{align}
    Q &= (M_A - M_B)c^2 &\quad \fysika{Fd2a} \label{freed}
\end{align}

\eqref{freed} gäller för reaktioner på formen $A \longrightarrow B$.

\subsection*{Lösning}

I \eqref{eq:csdecay1} sker endast ett $\beta^-$-sönderfall (inre konversion).
Den frigjorda energin omvandlas till kinetisk energi som fördelas mellan de
resulterande partiklarna. Den teoretiska maxenergin för elektronen är då
bariumatomen och antineutrinon får rörelseenergi godtyckligt nära noll, alltså
får elektronen all frigjord energi $Q$, som bestäms av \eqref{freed}:
%
\begin{equation}
    Q = (M_{Cs} - M_{Ba} - M_{\bar{\nu}_e})c^2 \approx (M_{Cs} - M_{Ba})c^2 \approx \qty{1.17554556}{\MeV}
\end{equation}

Notera att elektronmassan är medräknad i bariumatomens nuklidmassa, detta
eftersom den resulterande bariumatomen är joniserad medan tabellvärdet är
för en oladdad atom. Notera även att även antineutrinomassan räknas bort då
dess viloenergi, och därmed massa, är försumbar \fysika{Tk2}.

I \eqref{eq:csdecay2} gäller samma princip, men energin $E_\gamma$ förblir i
dotterkärnan efter det första sönderfallet. Den maximala energin $E_m$ till
elektronen fås då som:
%
\begin{equation}
    E_m = Q - E_\gamma \approx \qty{0.513889}{\MeV}
\end{equation}

Resultat: \qty{1.17554556}{\MeV} respektive \qty{0.513889}{\MeV}.
