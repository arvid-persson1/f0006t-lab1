\section{Metod} \label{sec:method}

\subsection{Mätning av radioaktivitet i livsmedel}

\begin{itemize}
    \item En scintillationsdetektor av modell GDM20 användes för att registrera
    gammastrålningen.

    \item En behållare innehållande provet placerades i detektorn och
    spektrumet registrerades genom datorprogrammet Windas.

    \item Bakgrundsstrålning mättes och subtraherades från resultaten.
\end{itemize}

Bakgrundsstrålningen mättes i förväg genom samma process som ovan, utan prov.
Detta innebär att basvärdena räknar med inte bara ``typisk'' bakgrundsstrålning
utan även den från lokalen och utrustningen. Efter subtraktion med dessa värden
förväntas alltså teoretiskt sett bara provets värden finnas kvar. Vissa fel
kommer från slumpen som följd av de låga mättiderna (se
avsnitt~\ref{sec:measurements}).

\subsection{Analys av betaspektrum från Cs-137}

\begin{itemize}
    \item För att registrera betastrålningen användes en halvledardetektor.

    \item Energispektrumet analyserades, och topparna för elektroner från inre 
    konversion identifierades.

    \item Värdena kalibrerades efter kända energivärden för den ena toppen för
    att kunna uppskatta mätningarnas precision på den andra.
\end{itemize}

% Vad är det här för något?
% v v v

Därefter, för att bestämma den maximala energi för betapartiklarna utfördes en extrapolering av spektrumets högra kant enligt ekvation:

\begin{equation}
E_{max} \approx \frac{dN}{dE} = 0 \text{\parencite{fysika}}
\end{equation}

där $E_{max}$ är den maximala energin \parencite{yf}.
