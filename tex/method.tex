\section{Metod} Laborationen bestod av två delar:

\subsection{Mätning av radioaktivitet i livsmedel}

\begin{itemize}

\item En scintillationsdetektor (GDM20) användes för att registrera gammastrålningen.

\item En liten behållare med svamp från Kiev, 2002, placerades i detektorn och, med hjälp av programmet Windas, deras spektra analyserades.

\item Bakgrundsstrålning mättes och subtraherades från resultaten.

\end{itemize}

För att bestämma aktiviteten i provet så användes följande beräkning:

\begin{equation}
A = \frac{N}{t} \cdot \frac{1}{K} \text{\parencite{fysika}}
\end{equation}

där $A$ är aktiviteten (Bq), $t$ är mättiden, $N$ är antalet detekterade sönderfall och $K$ är detektorns effektivitet \parencite{knoll}.

\subsection{Analys av betaspektrum från Cs-137}

\begin{itemize}

\item För att registrera betastrålning från en Cs-$137$-källa användes enn halvledardetektor.

\item En analys av Energispektrumet genomfördes, där topparna för elektroner från inre konversion identifierades.

\item En kalibrering med kända energivärden ufördes för att de korrekta mätvärden säkerställs.

\end{itemize}

Därefter, för att bestämma den maximala energi för betapartiklarna utfördes en extrapolering av spektrumets högra kant enligt ekvation:

\begin{equation}
E_{max} \approx \frac{dN}{dE} = 0 \text{\parencite{fysika}}
\end{equation}

där $E_{max}$ är den maximala energin \parencite{yf}.
