\section{Metod} \label{sec:method}

Bakgrundsstrålningen mättes i förväg genom samma process och utrustning som
beskrivs medan, utan prov. Detta innebär att basvärdena räknar med inte bara
``typisk'' bakgrundsstrålning utan även den från lokalen och utrustningen.
Efter subtraktion med dessa värden förväntas alltså teoretiskt sett bara
provets värden finnas kvar. Vissa fel kommer från slumpen som följd av de låga
mättiderna (se avsnitt~\ref{sec:measurements}).

Datorprogrammet Windas användes både för att både registrera och visualisera
pulserna, samt att beräkna fördelningsförhållandena.

\subsection{Mätning av radioaktivitet i livsmedel}

\begin{itemize}
    \item En scintillationsdetektor av modell GDM20 användes för att registrera
    gammastrålningen.

    \item En behållare innehållande provet placerades i detektorn och
    spektrumet registrerades.

    \item Bakgrundsstrålning subtraherades från resultaten.
\end{itemize}

\subsection{Analys av betaspektrum från Cs-137}

\begin{itemize}
    \item För att registrera betastrålningen användes en halvledardetektor.

    \item Bakgrundsstrålning subtraherades från resultaten.

    \item Energispektrumet analyserades, och topparna för elektroner från inre 
    konversion identifierades.

    \item Värdena kalibrerades efter kända energivärden för den ena toppen för
    att kunna uppskatta mätningarnas precision på den andra.
\end{itemize}
