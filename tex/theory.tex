\section{Teori} \label{sec:theory}

Radioaktivt sönderfall är en slumpmässig process där instabila atomkärnor omvandlas till ett mer stabilt tillstånd genom att frigöra energi i form av joniserande strålning. Det finns tre huvudtyper av radioaktivt sönderfall: alfasönderfall, beta-sönderfall och gammastrålning. \parencite{yf}
I denna laboration ligger fokus på beta-sönderfall och gammastrålning.

Söderfall av Cesium-137 ($^{137}_{55}\text{Cs}$) sker genom betaminus-sönderfall enligt reaktionen:

\begin{equation}
    ^{137}_{55}\text{Cs} \to ^{137m}_{56}\text{Ba} + \beta^- + \bar{\nu}_e 
\end{equation}

Dotterkärnan barium-137m ($^{137m}\text{Ba}$) är exciterad och avger gammastrålning på $0,6617$ MeV enligt:

\begin{equation}
^{137m}{56}\text{Ba} \to ^{137}{56}\text{Ba} + \gamma 
\end{equation}

\begin{group}
\noindent
    Gammastrålning är högenergi elektromagnetisk strålning som uppstår när atomkärnor övergår från ett exciterat tillstånd till ett grundtillstånd. Till skillnad från alfa- och betastrålning är gammastrålning oladdad och kan penetrera material som papper, plast och i viss mån bly \parencite{gilmore}.
\end{group}

\indent
Kalium-40 ($^{40}\text{K}$) är en annan radioaktiv isotop som ofta förekommer i livsmedel. Denna naturliga isotop bidrar till den naturliga bakgrundsstrålningen och kan sönderfalla på två sätt. \parencite{krane}

1. Betaminus-sönderfall:
\begin{equation}
^{40}{19}\text{K} \to ^{40}{20}\text{Ca} + \beta^- + \bar{\nu}_e 
\end{equation}

2. Elektroninfångning:
\begin{equation}
^{40}{19}\text{K} + e^- \to ^{40}{18}\text{Ar} + \nu_e 
\end{equation}

För att analysera dessa sönderfall används två huvudsakliga detektortyper:
\begin{itemize}
\item \textbf{Scintillationsdetektor:} Består av en kristall som avger ljus när den träffas av gammastrålning. Ljuset förstärks av en fotomultiplikator och omvandlas till en elektrisk signal \parencite{knoll}.
\item \textbf{Halvledardetektor:} Registrerar betastrålning genom att mäta de laddningsbärare som genereras när en betapartikel träffar en halvledarkomponent \parencite{gilmore}.
\end{itemize}
