\section{Teori} \label{sec:theory}

Radioaktivt sönderfall är en slumpmässig process där instabila atomkärnor
omvandlas till ett mer stabilt tillstånd genom att frigöra energi i form av
joniserande strålning. Det finns tre huvudtyper av radioaktivt sönderfall:
$\alpha$-, $\beta$- och $\gamma$-sönderfall \parencite{yf}. I denna laboration
är fokus på $\beta^-$-sönderfall (en variant av $\beta$-sönderfall) och
$\gamma$-sönderfall (även känd som gammastrålning).

Gammastrålning är högenergisk elektromagnetisk strålning som uppstår när
atomkärnor övergår från ett exciterat tillstånd till ett grundtillstånd. Till
skillnad från alfa- och betastrålning är gammastrålning oladdad och kan
penetrera material som papper, plast och i viss mån bly
\parencite{spectrometry}.

Söderfall av Cesium-137 ($^{137}_{55}\text{Cs}$) sker genom
$\beta^-$-sönderfall enligt reaktionerna
%
\begin{align}
    \ce{^137_55Cs} &\xrightarrow{\qty{5.6}{\percent}}  \ce{^137_56Ba} + e^- + \bar{\nu}_e,   \label{eq:csdecay1}, \\
    \ce{^137_55Cs} &\xrightarrow{\qty{94.4}{\percent}} \ce{^{137m}_56Ba} + e^- + \bar{\nu}_e \label{eq:csdecay2}.
\end{align}
%
Dotterkärnan Ba-137m ($^{137m}\text{Ba}$) är exciterad (metastabil) och
sönderfaller enligt reaktionerna
%
\begin{align}
    \ce{^{137m}_56Ba} \xrightarrow{\qty{85.1}{\percent}} \ce{^137_56Ba} + \gamma                     \label{eq:badecay1}, \\
    \ce{^{137m}_56Ba} + e^- \xrightarrow{\qty{85.1}{\percent}} \ce{^137_56Ba} + e^-_\text{emitterad} \label{eq:badecay2},
\end{align}
%
där gammaenergin $E_\gamma \approx \qty{0.661657}{\MeV}$.

Sönderfall av Kalium-40 ($^{40}\text{K}$) sker på ett av två sätt
\parencite{nuclear}~\parencite{instructions}:
%
\begin{enumerate}
    \item Genom $\beta^-$-sönderfall:
    %
    \begin{equation}
        \ce{^40_19K} \xrightarrow{\qty{89.33}{\percent}} \ce{^40_20Ca} + \beta^- + \bar{\nu}_e \label{eq:kdecay1}
    \end{equation}

    \item Genom elektroninfångning:
    %
    \begin{equation}
        \ce{^40_19K} \xrightarrow{\qty{10.67}{\percent}} \ce{^40_18Ar} + \bar{\nu}_e           \label{eq:kdecay2}
    \end{equation}
\end{enumerate}

För att analysera dessa sönderfall används två huvudsakliga detektortyper:
%
\begin{itemize}
    \item Scintillationsdetektorer består av en kristall som avger
    ljus när den träffas av gammastrålning. Ljuset förstärks av en
    fotomultiplikator och omvandlas till en elektrisk signal
    \parencite{radiation}.

    \item Halvledardetektorer registrerar betastrålning genom att mäta
    de laddningsbärare som genereras när en betapartikel träffar en
    halvledarkomponent \parencite{spectrometry}.
\end{itemize}
