\subsection{Slutsatser} \label{sec:conclusions}

Mätningar av livsmedelsproverna visar de aktivitetsnivåer som Livsmedelsverket
\parencite{livsmedelsverket} förväntar sig för marktäcket i det provtagna
området. Felkällor till mätvärdenas precision diskuteras ingående i
avsnitt~\ref{sec:method}~och~\ref{sec:analysis}.

Seltinet visade på en aktivitet per massa på \qty{6.3e3}{\becquerel\per\kg}
(se avsnitt~\ref{sec:activity}). Teoretisk beräkning visar en aktivitet på
\qty{4.3e2}{\becquerel} hos \qty{0.066}{\kg} Seltin, vilket ger en aktivitet
per massa på \qty{6.5e3}{\becquerel\per\kg}. Värdena överensstämmer med god
precision.

Livsmedelsverket har fastställt ett gränsvärde på \qty{300}{\becquerel\per\kg} 
som högsta tillåtna Cesiumhalt i livsmdel som säljs förutom ren- och viltkött,
insjöfisk, bär, svamp och nötter där gränsen ligger på
\qty{1500}{\becquerel\per\kg} \parencite{livsmedelsverket}. Vid jämförelse
med provernas värden framgår att aktiviteten långt överskrider de tillåtna
värdena och därmed klassas som osäkra för (regelbunden) konsumtion.

Två toppar observerades i betaspektrumet, motsvarande elektronerna från inre
konversion i K- respektive L-skalet (se avsnitt~\ref{sec:cesium}). Datan
kalibrerades baserat på beräknat värde för K-toppen (se
bilaga~\ref{sec:conenergy}), vilket gav ett uppmätt värde på \qty{0.657}{\MeV}
för L-toppen, nära beräknat värde på \qty{0.6564}{\MeV}.

Den maximala energin för $\beta$-sönderfallet av Cs-137 till Ba-137 i exciterat
tillstånd beräknades till \qty{0.59}{\MeV}, jämfört med det teoretiska värdet
\qty{0.51}{\MeV}, resulterande i en felmarginal på \qty{14}{\percent}. Eftersom
detta fel är avsevärt större än andra fel beror det rimligen på metod snarare
än data (felkällor diskuteras i avsnitt~\ref{sec:betamax}).

Sammanfattningsvis gav resultaten en fascinerande inblick i hur
radioaktiviteten från Tjernobylolyckan fortfarande påverkar vår vardag, nästan
\num{40} år senare. Särskilt intressant är de olika beteenden isotoper såsom
I-131 och Cs-137 uppvisar i miljön och näringskedjan, vilket i sin tur påverkar
både mätmetoder och hälsoeffekter.
