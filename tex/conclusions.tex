\subsection{Slutsatser} \label{sec:conclusions}

Genom analys av mätresultaten från de olika proverna har vi kunnat dra flera viktiga slutsatser. 

\indent

Mätningar av livsmedelsproverna visar de aktivitetsnivåer som Livsmedelsverket \parencite{livsmedelsverket} förväntar sig för marktäcket i det provtagna området. Mätningsnoggrannheten påverkas främst av mättiden Ju längre mättid, desto högre statistisk säkerhet och desto mer tillförlitligt resultat.

När det kommer till Seltin så uppvidsades en karakteristisk topp vid $1,455$ MeV från saltprovet, som då väl motsvarar K-$40$:s kända gammastrålning ($1,461$ MeV).
Dessa mätningar visade dock mer brus än svampprovet, vilket kan huvudsaknligen bero på den lägre detekterade pulsfrekvensen.

\indent

När vi jämför med Livsmedelsverkets gränsvärden visar det analyserade livsmedelsprovet sig ligga under de tillåtna nivåerna för livsmedelsförsäljning. Detta innebär att produkten är säker för både konsumtion och handel enligt gällande föreskrifter.


Två distinkta toppar observerades vid analys av $\beta$-strålningen från Cs-137-staven, toppar som var vid 0,624 MeV respektive 0,657 MeV. Dessa experimentella värdena visar att elektronerna kan nå energinivåer upp till 1,2 MeV, vilket överstämde väl med de energinivåerna som förvändades för elektronerna från inre konversion i K-skalen och L-skalen.
Dessutom så stämmer den uppmätta energin för L-toppen väl överens med det teoretiskt beräknade värdet, med en avvikelse på mindre än 2$\%$. Detta stärker mätinstrumentets pålitlighet, särskilt med tanke på att K-toppen användes för energikalibrering.

\indent

Dock så påverkades precisionen i mätresultaten betydligt av mättiden; längre mättider ledde till en högre statistisk säkerhet och mindre spridning i datan. Detta var särskilt märkbart vid analysen av de svagare topparna i spektrumet.

Det krävdes även korrektion vid aktivitetsberäkningarna, då detektrons känslighet varierade med strålningens energi. I efterhand kan vi resonera att, för mer exakt resultat, skulle längre mättider kunna samlas in, särkild för saltprovet som har en längre aktivitet.


\indent

Sammanfattningsvis så ger resultaten en fascinerande inblick i hur radioaktiviteten från Tjernobylolyckan fortfarande påverkar vår vardag, nästan $40$ år senare. Särskilt intressant är även de olika beteenden som isotoper som I-$131$ och Cs-$137$ uppvisar i miljön och näringskedjan, vilket i sin tur påverkar både mätmetoder och hälsoeffekter. 

Den höga penetrationsförmågan hos $\gamma$-strålning, jämfört med $\beta$-strålning i luft, förklarar varför mätningar utförda med flygplan efter Tjernobylolyckan fokuserade på $\gamma$-strålningen. Samtidigt ger de konstanta $\beta$-spektrumen viktiga insikter i hur partiklarnas energi fördelar sig vid radioaktivt sönderfall.
