\section{Maximal elektronhastighet vid Cs-137-sönderfall} \label{sec:speed}

Elektronernas maxhastighet för $\beta^-$-sönderfall av Cs-137 ska beräknas
uttryckt i ljushastigheten.

\subsection{Konstanter}

\begin{align*}
    & \text{Ljusets hastighet}     & c   = \qty{299792458}{\m\per\s}                  &\quad \text{(\cite{fysika}, Te)}   \\
    & \text{Elektronmassa}         & m_e = \qty{0.510998950}{\MeV\per\clight\squared} &\quad \text{(\cite{fysika}, Te)}   \\
    & \text{Maximal rörelseenergi} & K   = \qty{1.17554556}{\MeV}                     &\quad \text{(Bilaga~\ref{sec:energy})}
\end{align*}

\subsection{Formler}

\begin{align}
    \gamma &= \frac{1}{\sqrt{1 - v^2/c^2}} &\quad \text{(\cite{fysika}, Fd1)} \label{eq:lorentz} \\
         K &= mc^2(\gamma - 1)             &\quad \text{(\cite{fysika}, Fd1)} \label{eq:kinetic}
\end{align}

\subsection{Lösning}

Av \eqref{eq:kinetic} fås Lorentzfaktorn $\gamma$:

\begin{equation}
    K = m_e c^2(\gamma - 1) \iff \gamma = \frac{K}{m_e c^2} + 1 \label{eq:lorentz2}
\end{equation}

Av \eqref{eq:lorentz} fås hastigheten $v$:

\begin{equation}
    \gamma = \frac{1}{\sqrt{1 - v^2/c^2}} \iff v = c \sqrt{1 - \frac{1}{\gamma^2}}~\text{(\unit{\m\per\s})} \label{eq:speed}
\end{equation}

\eqref{eq:lorentz2} i \eqref{eq:lorentz} ger:

\begin{equation}
    v = c \sqrt{1 - \left( \frac{K}{m_e c^2} + 1 \right)^2}~\text{(\unit{\m\per\s})}
\end{equation}

$v$ söks uttryckt i $c$:

\begin{equation}
    \frac{v}{c} \approx \num{0.952995087}
\end{equation}

Resultat: \qty{0.952995087}{\clight}
