\section{Diskussion och slutsatser}

\subsection{Strålningsmätning från flygplan}

I \cite{penetration} redovisas experimentella resultat för penetrationsförmågan
hos olika typer av aktivitet. $\beta$-partiklar interagerar genom
Coulombkrafter med elektroner i omgivningen. Detta leder till att de kan
stoppas helt av relativt lite materia. Givet energin av en betapartikel kan
ett teoretiskt maximalt penetrationsdjup finnas.

Experimenten utfördes på prover av Sr/Y-90 som har speciellt höga energinivåer
på maximalt \qty{2.27}{\MeV}, genomsnittligt \qty{1.13}{\MeV}. Dessa nivåer ger
en penetrationsförmåga i luft (densitet \qty{1.2}{\mg\per\cm\cubed}) på
\qty{8.8}{\m} respektive \qty{3.8}{\m}. Det bör noteras att betaenerginivåerna
för Cs-137 är betydligt lägre \parencite{fysika} och att penetrationsförmågan
därmed är betydligt lägre.

Bristen på laddning i $\gamma$-strålning innebär att interaktionerna med
materia sker på helt andra sätt, vilket gör strålningen mer penetrerande. Då
interaktionerna är av statistisk natur kan man inte med någon absorbent
förväntas stoppa alla $\gamma$-strålning utan det är mer relevant att diskutera
t.ex. ``halveringsdjup'' i olika material.

För luft är absorbtionsförmågan relativt låg, alltså förväntas en stor andel
gammastrålning från marken kunna betraktas från ett lågflygande flygplan. Detta
är varför det efter Tjernobylolyckan endast var $\gamma$-strålning som mättes
med hjälp av flygplan.

\subsection{I-131 i Sverige}

Också i samband med Tjernobylolyckan spreds I-131 över Europa. Isotopen tar sig
främst till människor genom mjölk, som i sin tur kommer från kor som har ätit
gräs där regnet har varit kontaminerat.

I-131 har dock en kort halveringstid på \qty{8.025}{\day} \parencite{fysika},
och olyckan skedde i april då kor i Sverige än inte betar utomhus, alltså hade
aktiviteten hunnit sjunka till en lägre nivå innan den nådde toppen av
näringskedjan. Det är också därför det drabbade södra Europa hårdare, där djur
betar tidigare på året.

\subsection{Faran med I-131}

Det som utmärker I-131 som en hälsorisk är hur det attackerar sköldkörteln.
Sköldkörteln använder jod för att producera livsnödvändiga tyreoideahormoner,
vilket leder till att dessa isotoper tenderar att samlas där. \cite{iodine}
visar på en stark korrelation mellan joniserande strålning och
sköldkörtelcancer. Dessa risker kan mitigeras något genom jodtabletter eftersom
joden i dem mättar körteln.

\subsection{Kontinuerliga $\beta$-spektrum}

Betrakta reaktionen för $\beta$-sönderfall:

\begin{equation}
    n \longrightarrow p + \beta^- + \bar{\nu}_e \quad \text{\parencite{yf}}
\end{equation}

Märk att tre partiklar bildas i processen. Detta medför, till skillnad från
fallet med två partiklar, att rörelsemängd och energi fritt kan fördelas
mellan de tre utan att överträda lagarna om bevaring av rörelsemängd och
energi. I praktiken innebär det att den frigjorda energin delas av elektronen
($\beta^-$-partikeln) och antineutrinon, alltså uppmätt ett kontinuerligt
energispektrum vid observation av $\beta^-$-partiklar.
