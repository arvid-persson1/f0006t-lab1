\section{Diskussion och slutsatser}

\subsection{Strålningsmätning från flygplan}

I \cite{penetration} redovisas experimentella resultat för penetrationsförmågan
hos olika typer av aktivitet. $\beta$-partiklar interagerar genom
Coulombkrafter med elektroner i omgivningen. Detta leder till att de kan
stoppas helt av relativt lite materia. Givet energin av en betapartikel kan
ett teoretiskt maximalt penetrationsdjup finnas.

Experimenten utfördes på prover av Sr/Y-90 som har speciellt höga energinivåer
på maximalt \qty{2.27}{\MeV}, genomsnittligt \qty{1.13}{\MeV}. Dessa nivåer ger
en penetrationsförmåga i luft (densitet \qty{1.2}{\mg\per\cm\cubed}) på
\qty{8.8}{\m} respektive \qty{3.8}{\m}. Det bör noteras att betaenerginivåerna
för Cs-137 är betydligt lägre \parencite{fysika} och att penetrationsförmågan
därmed är betydligt lägre.

Bristen på laddning i $\gamma$-strålning innebär att interaktionerna med
materia sker på helt andra sätt, vilket gör strålningen mer penetrerande. Då
interaktionerna är av statistisk natur kan man inte med någon absorbent
förväntas stoppa alla $\gamma$-strålning utan det är mer relevant att diskutera
t.ex. ``halveringsdjup'' i olika material.

För luft är absorbtionsförmågan relativt låg, alltså förväntas en stor andel
gammastrålning från marken kunna betraktas från ett lågflygande flygplan. Detta
är varför det efter Tjernobylolyckan endast var $\gamma$-strålning som mättes
med hjälp av flygplan.
